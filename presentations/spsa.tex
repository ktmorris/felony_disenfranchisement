% Options for packages loaded elsewhere
\PassOptionsToPackage{unicode}{hyperref}
\PassOptionsToPackage{hyphens}{url}
%
\documentclass[
  ignorenonframetext,
]{beamer}
\usepackage{pgfpages}
\setbeamertemplate{caption}[numbered]
\setbeamertemplate{caption label separator}{: }
\setbeamercolor{caption name}{fg=normal text.fg}
\beamertemplatenavigationsymbolsempty
% Prevent slide breaks in the middle of a paragraph
\widowpenalties 1 10000
\raggedbottom
\setbeamertemplate{part page}{
  \centering
  \begin{beamercolorbox}[sep=16pt,center]{part title}
    \usebeamerfont{part title}\insertpart\par
  \end{beamercolorbox}
}
\setbeamertemplate{section page}{
  \centering
  \begin{beamercolorbox}[sep=12pt,center]{part title}
    \usebeamerfont{section title}\insertsection\par
  \end{beamercolorbox}
}
\setbeamertemplate{subsection page}{
  \centering
  \begin{beamercolorbox}[sep=8pt,center]{part title}
    \usebeamerfont{subsection title}\insertsubsection\par
  \end{beamercolorbox}
}
\AtBeginPart{
  \frame{\partpage}
}
\AtBeginSection{
  \ifbibliography
  \else
    \frame{\sectionpage}
  \fi
}
\AtBeginSubsection{
  \frame{\subsectionpage}
}
\usepackage{lmodern}
\usepackage{amssymb,amsmath}
\usepackage{ifxetex,ifluatex}
\ifnum 0\ifxetex 1\fi\ifluatex 1\fi=0 % if pdftex
  \usepackage[T1]{fontenc}
  \usepackage[utf8]{inputenc}
  \usepackage{textcomp} % provide euro and other symbols
\else % if luatex or xetex
  \usepackage{unicode-math}
  \defaultfontfeatures{Scale=MatchLowercase}
  \defaultfontfeatures[\rmfamily]{Ligatures=TeX,Scale=1}
\fi
\usetheme[]{Berlin}
% Use upquote if available, for straight quotes in verbatim environments
\IfFileExists{upquote.sty}{\usepackage{upquote}}{}
\IfFileExists{microtype.sty}{% use microtype if available
  \usepackage[]{microtype}
  \UseMicrotypeSet[protrusion]{basicmath} % disable protrusion for tt fonts
}{}
\makeatletter
\@ifundefined{KOMAClassName}{% if non-KOMA class
  \IfFileExists{parskip.sty}{%
    \usepackage{parskip}
  }{% else
    \setlength{\parindent}{0pt}
    \setlength{\parskip}{6pt plus 2pt minus 1pt}}
}{% if KOMA class
  \KOMAoptions{parskip=half}}
\makeatother
\usepackage{xcolor}
\IfFileExists{xurl.sty}{\usepackage{xurl}}{} % add URL line breaks if available
\IfFileExists{bookmark.sty}{\usepackage{bookmark}}{\usepackage{hyperref}}
\hypersetup{
  pdftitle={Felony Disenfranchisement and Neighborhood Turnout},
  pdfauthor={Kevin Morris},
  hidelinks,
  pdfcreator={LaTeX via pandoc}}
\urlstyle{same} % disable monospaced font for URLs
\newif\ifbibliography
\setlength{\emergencystretch}{3em} % prevent overfull lines
\providecommand{\tightlist}{%
  \setlength{\itemsep}{0pt}\setlength{\parskip}{0pt}}
\setcounter{secnumdepth}{-\maxdimen} % remove section numbering
\newlength{\cslhangindent}
\setlength{\cslhangindent}{1.5em}
\newenvironment{cslreferences}%
  {\setlength{\parindent}{0pt}%
  \everypar{\setlength{\hangindent}{\cslhangindent}}\ignorespaces}%
  {\par}

\title{Felony Disenfranchisement and Neighborhood Turnout}
\subtitle{The Case of New York City}
\author{Kevin Morris}
\date{Southern Political Science Association, 2020}
\institute{Brennan Center for Justice}

\begin{document}
\frame{\titlepage}
\begin{abstract}
\href{mailto:kevin.morris@nyu.edu}{\nolinkurl{kevin.morris@nyu.edu}}
\end{abstract}

\begin{frame}{Outline}
\protect\hypertarget{outline}{}
\begin{itemize}[<+->]
\tightlist
\item
  Introducing space into the conversation
\end{itemize}

\begin{itemize}[<+->]
\tightlist
\item
  Redefining ``lost voters''
\end{itemize}

\begin{itemize}[<+->]
\tightlist
\item
  Identifying neighborhoods with lost voters
\end{itemize}

\begin{itemize}[<+->]
\tightlist
\item
  Testing turnout effects
\end{itemize}
\end{frame}

\begin{frame}{Introducing Space}
\protect\hypertarget{introducing-space}{}
\begin{itemize}[<+->]
\tightlist
\item
  Most of the existing literature on felony disenfranchisement and
  indirect turnout effects looks for effects at the state level (Miles
  2004; Bowers and Preuhs 2009; King and Erickson 2016)
\end{itemize}

\begin{itemize}[<+->]
\tightlist
\item
  But we \emph{know} that incarceration patterns aren't uniformly
  distributed throughout a given state (see, for instance, Gelman,
  Fagan, and Kiss 2007)
\end{itemize}

\begin{itemize}[<+->]
\tightlist
\item
  Turnout effects are likely to be socially mediated - and therefore
  show up in the neighborhoods where disenfranchised individuals live
  (e.g.~Foladare 1968; Huckfeldt 1979; Cho, Gimpel, and Dyck 2006)
\end{itemize}
\end{frame}

\begin{frame}{Redefining Lost Voters}
\protect\hypertarget{redefining-lost-voters}{}
\begin{itemize}[<+->]
\tightlist
\item
  Much of the existing literature cannot identify individuals who
  \emph{would have voted} if not for their disenfranchisement (Burch
  2013)
\end{itemize}

\begin{itemize}[<+->]
\tightlist
\item
  This undermines our ability to distinguish the causal effects of
  incarceration / probation from the effects of disenfranchisement
\end{itemize}

\begin{itemize}[<+->]
\tightlist
\item
  This is especially likely given the low turnout propensity among the
  formerly disenfranchised (e.g.~Meredith and Morse 2015; Gerber et al.
  2017; White 2019)
\end{itemize}
\end{frame}

\begin{frame}{This Project Addresses Both Problems}
\protect\hypertarget{this-project-addresses-both-problems}{}
\begin{itemize}[<+->]
\tightlist
\item
  Administrative data from the NYS DOC allows us to conduct
  individual-level analysis
\end{itemize}

\begin{itemize}[<+->]
\tightlist
\item
  Purge records in the NYS voter file allow us to construct vote
  histories for disenfranchised individuals
\end{itemize}

\begin{itemize}[<+->]
\tightlist
\item
  Lost voters are all formally disenfranchised individuals who have
  voted in the past 10 years
\end{itemize}
\end{frame}

\begin{frame}{Identifying Neighborhoods with Lost Voters}
\protect\hypertarget{identifying-neighborhoods-with-lost-voters}{}
\includegraphics[width=1\linewidth]{../temp/lost_voters_map}
\end{frame}

\begin{frame}{Methodology}
\protect\hypertarget{methodology}{}
\begin{itemize}
\item
  A genetic match algorithm (Sekhon 2011) is used to match treated
  census block groups to untreated ones.
\item
  Each block group is matched to 30 untreated block groups; matches are
  done with replacement.
\end{itemize}
\end{frame}

\begin{frame}{Matching Results}
\protect\hypertarget{matching-results}{}
\begin{center}\includegraphics[width=0.49\linewidth,height=0.49\textheight]{../temp/perc_white_spsa} \includegraphics[width=0.49\linewidth,height=0.49\textheight]{../temp/perc_black_spsa} \end{center}
\end{frame}

\begin{frame}{Matching Results}
\protect\hypertarget{matching-results-1}{}
\begin{center}\includegraphics[width=0.49\linewidth,height=0.49\textheight]{../temp/perc_latino_spsa} \includegraphics[width=0.49\linewidth,height=0.49\textheight]{../temp/income_spsa} \end{center}
\end{frame}

\begin{frame}{Testing Turnout Effects}
\protect\hypertarget{testing-turnout-effects}{}
\begin{center}\includegraphics[width=0.85\linewidth,height=0.85\textheight]{../temp/coef_plot1} \end{center}
\end{frame}

\begin{frame}{Testing Turnout Effects}
\protect\hypertarget{testing-turnout-effects-1}{}
\begin{center}\includegraphics[width=0.85\linewidth,height=0.85\textheight]{../temp/coef_plot2} \end{center}
\end{frame}

\begin{frame}{Testing Turnout Effects}
\protect\hypertarget{testing-turnout-effects-2}{}
\begin{center}\includegraphics[width=0.85\linewidth,height=0.85\textheight]{../temp/coef_plot3b} \end{center}
\end{frame}

\begin{frame}{Testing Turnout Effects}
\protect\hypertarget{testing-turnout-effects-3}{}
\includegraphics[width=1\linewidth]{../output/dep_map}
\end{frame}

\begin{frame}{Conclusions}
\protect\hypertarget{conclusions}{}
\begin{itemize}[<+->]
\tightlist
\item
  The indirect turnout effects identified in past research appear to be
  geographically concentrated where the lost voters live.
\end{itemize}

\begin{itemize}[<+->]
\tightlist
\item
  Hajnal (2009) and others have demonstrated that turnout differentials
  can have real political implications, especially in low-turnout
  contests.
\end{itemize}

\begin{itemize}[<+->]
\tightlist
\item
  Felony disenfranchisement undermines Black political representation at
  the local level.
\end{itemize}
\end{frame}

\begin{frame}{We Made It!}
\protect\hypertarget{we-made-it}{}
Thanks!

\href{mailto:kevin.morris@nyu.edu}{\nolinkurl{kevin.morris@nyu.edu}}
\end{frame}

\begin{frame}[allowframebreaks,allowframebreaks]{References}
\protect\hypertarget{references}{}
\hypertarget{refs}{}
\begin{cslreferences}
\leavevmode\hypertarget{ref-Bowers2009}{}%
Bowers, Melanie, and Robert R. Preuhs. 2009. ``Collateral Consequences
of a Collateral Penalty: The Negative Effect of Felon Disenfranchisement
Laws on the Political Participation of Nonfelons.'' \emph{Social Science
Quarterly} 90 (3): 722--43.
\url{https://doi.org/10.1111/j.1540-6237.2009.00640.x}.

\leavevmode\hypertarget{ref-Burch2013}{}%
Burch, Traci. 2013. ``Effects of Imprisonment and Community Supervision
on Neighborhood Political Participation in North Carolina.'' Edited by
Christopher Wildeman, Jacob S. Hacker, and Vesla M. Weaver. \emph{The
ANNALS of the American Academy of Political and Social Science} 651 (1):
184--201. \url{https://doi.org/10.1177/0002716213503093}.

\leavevmode\hypertarget{ref-Cho2006}{}%
Cho, Wendy K. Tam, James G. Gimpel, and Joshua J. Dyck. 2006.
``Residential Concentration, Political Socialization, and Voter
Turnout.'' \emph{The Journal of Politics} 68 (1): 156--67.
\url{https://doi.org/10.1111/j.1468-2508.2006.00377.x}.

\leavevmode\hypertarget{ref-Foladare1968}{}%
Foladare, Irving S. 1968. ``The Effect of Neighborhood on Voting
Behavior.'' \emph{Political Science Quarterly} 83 (4): 516.
\url{https://doi.org/10.2307/2146812}.

\leavevmode\hypertarget{ref-Gelman2007}{}%
Gelman, Andrew, Jeffrey Fagan, and Alex Kiss. 2007. ``An Analysis of the
New York City Police Departments ``Stop-and-Frisk'' Policy in the
Context of Claims of Racial Bias.'' \emph{Journal of the American
Statistical Association} 102 (479): 813--23.
\url{https://doi.org/10.1198/016214506000001040}.

\leavevmode\hypertarget{ref-Gerber2017}{}%
Gerber, Alan S., Gregory A. Huber, Marc Meredith, Daniel R. Biggers, and
David J. Hendry. 2017. ``Does Incarceration Reduce Voting? Evidence
About the Political Consequences of Spending Time in Prison.'' \emph{The
Journal of Politics} 79 (4): 1130--46.
\url{https://doi.org/10.1086/692670}.

\leavevmode\hypertarget{ref-Hajnal2009}{}%
Hajnal, Zoltan. 2009. \emph{America's Uneven Democracy}. Cambridge
University Press. \url{https://doi.org/10.1017/cbo9780511800535}.

\leavevmode\hypertarget{ref-Huckfeldt1979}{}%
Huckfeldt, R. Robert. 1979. ``Political Participation and the
Neighborhood Social Context.'' \emph{American Journal of Political
Science} 23 (3): 579. \url{https://doi.org/10.2307/2111030}.

\leavevmode\hypertarget{ref-King2016}{}%
King, Bridgett A., and Laura Erickson. 2016. ``Disenfranchising the
Enfranchised.'' \emph{Journal of Black Studies} 47 (8): 799--821.
\url{https://doi.org/10.1177/0021934716659195}.

\leavevmode\hypertarget{ref-Meredith2015}{}%
Meredith, Marc, and Michael Morse. 2015. ``The Politics of the
Restoration of Ex-Felon Voting Rights: The Case of Iowa.''
\emph{Quarterly Journal of Political Science} 10 (1): 41--100.
\url{https://doi.org/10.1561/100.00013026}.

\leavevmode\hypertarget{ref-Miles2004}{}%
Miles, Thomas J. 2004. ``Felon Disenfranchisement and Voter Turnout.''
\emph{The Journal of Legal Studies} 33 (1): 85--129.
\url{https://doi.org/10.1086/381290}.

\leavevmode\hypertarget{ref-Sekhon2011}{}%
Sekhon, Jasjeet S. 2011. ``Multivariate and Propensity Score Matching
Software with Automated Balance Optimization: The Matching Package for
R.'' \emph{Journal of Statistical Software} 42 (7).
\url{https://doi.org/10.18637/jss.v042.i07}.

\leavevmode\hypertarget{ref-White2019}{}%
White, Ariel. 2019. ``Misdemeanor Disenfranchisement? The Demobilizing
Effects of Brief Jail Spells on Potential Voters.'' \emph{American
Political Science Review} 113 (2): 311--24.
\url{https://doi.org/10.1017/s000305541800093x}.
\end{cslreferences}
\end{frame}

\end{document}
